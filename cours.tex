\documentclass[11pt,a4paper]{article}
\usepackage[utf8]{inputenc}
\usepackage[french]{babel}
\usepackage[T1]{fontenc}
\usepackage{amsmath,amsfonts,amssymb}
\usepackage{geometry}
\usepackage{graphicx}
\usepackage{tikz}
\usepackage{pgfplots}
\usepackage{xcolor}
\usepackage{fancyhdr}
\usepackage{tcolorbox}
\usepackage{enumitem}

\geometry{margin=2.5cm}
\pgfplotsset{compat=1.18}

% Configuration des couleurs
\definecolor{bleuprof}{RGB}{0, 102, 204}
\definecolor{vertmath}{RGB}{34, 139, 34}
\definecolor{rougealerte}{RGB}{220, 20, 60}
\definecolor{grisfonce}{RGB}{64, 64, 64}

% Configuration des boîtes
\tcbuselibrary{skins}
\newtcolorbox{definition}{
	colback=blue!5!white,
	colframe=bleuprof,
	title=Définition,
	fonttitle=\bfseries,
	arc=2mm
}

\newtcolorbox{propriete}{
	colback=green!5!white,
	colframe=vertmath,
	title=Propriété,
	fonttitle=\bfseries,
	arc=2mm
}

\newtcolorbox{exemple}{
	colback=orange!5!white,
	colframe=orange!80!black,
	title=Exemple,
	fonttitle=\bfseries,
	arc=2mm
}

\newtcolorbox{attention}{
	colback=red!5!white,
	colframe=rougealerte,
	title=Attention,
	fonttitle=\bfseries,
	arc=2mm
}

\newtcolorbox{methode}{
	colback=purple!5!white,
	colframe=purple!80!black,
	title=Méthode,
	fonttitle=\bfseries,
	arc=2mm
}

\newtcolorbox{demonstration}{
	colback=gray!5!white,
	colframe=gray!80!black,
	title=Démonstration,
	fonttitle=\bfseries,
	arc=2mm
}

% En-tête et pied de page
\pagestyle{fancy}
\fancyhf{}
\fancyhead[L]{\textbf{Mathématiques - Première générale}}
\fancyhead[R]{\textbf{Chapitre 1}}
\fancyfoot[C]{\thepage}

\title{
	\vspace{-2cm}
	\Huge\textbf{\textcolor{bleuprof}{Chapitre 1}}\\
	\vspace{0.5cm}
	\LARGE\textbf{Fonction polynôme du second degré}\\
	\large\textit{Partie 1 : formes développée et factorisée. Racines.}
}
\author{}
\date{}

\begin{document}
	
	\maketitle
	\vspace{-1cm}
	
	\section{Introduction}
	
	\begin{tcolorbox}[colback=gray!10!white, colframe=grisfonce, title=Objectifs du chapitre, fonttitle=\bfseries]
		À la fin de ce chapitre, vous saurez :
		\begin{itemize}[label=$\bullet$]
			\item Reconnaître une fonction polynôme du second degré sous forme factorisée ou développée.
			\item Déterminer les racines d'une fonction polynôme du second degré donnée sous forme factorisée.
			\item Étudier le signe d'une fonction polynôme du second degré donnée sous forme factorisée
			\item Utiliser les relations entre coefficients et racines notamment quand on connaît une racine
		\end{itemize}
	\end{tcolorbox}
	
	\vspace{0.5cm}
	
	Les fonctions polynômes du second degré sont omniprésentes en mathématiques et dans de nombreux domaines d'application : physique (trajectoires de projectiles), économie (optimisation de profits), géométrie (calculs d'aires)...
	
	\section{Définitions et vocabulaire}
	
	\begin{definition}
		Une \textbf{fonction polynôme du second degré} est une fonction $f$ définie sur $\mathbb{R}$ par :
		$$f(x) = ax^2 + bx + c$$
		où $a$, $b$ et $c$ sont des nombres réels avec $a \neq 0$.
		
		On dit que cette fonction est donnée sous \textbf{forme développée réduite}.
	\end{definition}
	
	\begin{definition}
		Une fonction polynôme du second degré est donnée sous \textbf{forme factorisée} lorsqu'elle s'écrit :
		$$f(x) = a(x - \alpha)(x - \beta)$$
		où $a \neq 0$, et $\alpha$ et $\beta$ sont des nombres réels.
	\end{definition}
	
	\begin{exemple}
		\begin{itemize}
			\item $f(x) = 2(x - 1)(x + 3)$ est sous forme factorisée avec $a = 2$, $\alpha = 1$ et $\beta = -3$
			\item $g(x) = -\frac{1}{2}(x - 4)(x - 7)$ est sous forme factorisée avec $a = -\frac{1}{2}$, $\alpha = 4$ et $\beta = 7$
			\item $h(x) = 3(x + 2)^2$ peut s'écrire $h(x) = 3(x - (-2))(x - (-2))$ avec $\alpha = \beta = -2$
		\end{itemize}
	\end{exemple}
	
	\section{Racines d'une fonction polynôme du second degré}
	
	\begin{definition}
		Les \textbf{racines} (ou \textbf{zéros}) d'une fonction polynôme du second degré $f$ sont les valeurs de $x$ pour lesquelles $f(x) = 0$.
		
		Graphiquement, les racines correspondent aux abscisses des points d'intersection de la courbe représentative avec l'axe des abscisses.
	\end{definition}
	
	\begin{center}
		\begin{tikzpicture}
			\begin{axis}[
				axis lines = middle,
				xlabel = $x$,
				ylabel = $y$,
				domain = -2.5:6.5,
				samples = 100,
				grid = major,
				width = 12cm,
				height = 8cm,
				xmin = -2.5, xmax = 6.5,
				ymin = -2, ymax = 20
				]
				
				% Fonction f(x) = -2(x-5)(x+1) = -2(x^2 - 4x - 5) = -2x^2 + 8x + 10
				\addplot[blue, thick, smooth] {-2*x^2 + 8*x + 10};
				
				% Points d'intersection avec l'axe des x (racines)
				\addplot[red, only marks, mark=*, mark size=3pt] coordinates {(-1,0) (5,0)};
				
				% Sommet
				\addplot[green, only marks, mark=*, mark size=3pt] coordinates {(2,18)};
				
				% Axe de symétrie
				\addplot[purple, thick, dashed] coordinates {(2,-2) (2,20)};
				
				% Étiquettes
				\node[blue] at (axis cs:3.5,15) {$f(x) = -2(x-5)(x+1)$};
				\node[green] at (axis cs:2,16) {Sommet $(2; 18)$};
				\node[purple] at (axis cs:2.8,1) {Axe de symétrie : $x = 2$};
				
				% Flèches pointant vers les racines
				\draw[red, thick, ->] (axis cs:-0.5,3) -- (axis cs:-1,0.5);
				\draw[red, thick, ->] (axis cs:4.5,3) -- (axis cs:5,0.5);
				
			\end{axis}
		\end{tikzpicture}
	\end{center}
	
	\begin{attention}
		\textbf{Remarque importante admise pour le moment:} L'axe de symétrie de la parabole passe par le point d'abscisse égale à la moyenne des racines.
		
		Dans cet exemple : racines $-1$ et $5$, moyenne $= \frac{-1 + 5}{2} = 2$.
		
		L'axe de symétrie a donc pour équation $x = 2$, et c'est aussi l'abscisse du sommet.
	\end{attention}
	
	\subsection{Lecture directe des racines}
	
	\begin{propriete}
		Si une fonction polynôme du second degré est donnée sous forme factorisée :
		$$f(x) = a(x - \alpha)(x - \beta)$$
		alors les valeurs $\alpha$ et $\beta$ sont les racines de la fonction $f$.
		
		En effet : $f(\alpha) = a(\alpha - \alpha)(\alpha - \beta) = a \times 0 \times (\alpha - \beta) = 0$
		
		De même : $f(\beta) = a(\beta - \alpha)(\beta - \beta) = a \times (\beta - \alpha) \times 0 = 0$
		
		On admet de plus, pour le moment, que si une fonction du second degré $f$ définie sur $\mathbb{R}$ par :
		$$f(x) = ax^2 + bx + c$$
		admet deux racines $\alpha$ et $\beta$ alors on peut écrire :
		$$f(x) = a(x - \alpha)(x - \beta)$$
		
		
	\end{propriete}
	
	\begin{exemple}
		Déterminons les racines de $f(x) = -2(x - 5)(x + 1)$.
		
		La fonction est sous forme factorisée avec $a = -2$, $\alpha = 5$ et $\beta = -1$.
		
		Les racines sont donc $5$ et $-1$.
		
		\textbf{Vérification :}
		\begin{align*}
			f(5) &= -2(5 - 5)(5 + 1) = -2 \times 0 \times 6 = 0 \\
			f(-1) &= -2(-1 - 5)(-1 + 1) = -2 \times (-6) \times 0 = 0
		\end{align*}
	\end{exemple}
	
	\section{Étude du signe d'une fonction polynôme du second degré}
	
	\subsection{Rappel : Étude du signe d'une fonction affine}
	
	\begin{propriete}[Signe d'une fonction affine]
		Soit $f(x) = ax + b$ une fonction affine avec $a \neq 0$.
		
		La fonction $f$ s'annule pour $x = -\frac{b}{a}$.
		
		\textbf{Si $a > 0$ :}
		\begin{itemize}
			\item $f(x) < 0$ pour $x < -\frac{b}{a}$
			\item $f(x) = 0$ pour $x = -\frac{b}{a}$
			\item $f(x) > 0$ pour $x > -\frac{b}{a}$
		\end{itemize}
		
		\textbf{Si $a < 0$ :}
		\begin{itemize}
			\item $f(x) > 0$ pour $x < -\frac{b}{a}$
			\item $f(x) = 0$ pour $x = -\frac{b}{a}$
			\item $f(x) < 0$ pour $x > -\frac{b}{a}$
		\end{itemize}
	\end{propriete}
	
	\begin{exemple}
		Étudions le signe de $g(x) = 3x - 6$.
		
		On a $a = 3 > 0$ et $b = -6$.
		
		$g(x) = 0 \Leftrightarrow 3x - 6 = 0 \Leftrightarrow x = 2$
		
		Comme $a > 0$ :
		\begin{itemize}
			\item $g(x) < 0$ pour $x < 2$
			\item $g(x) = 0$ pour $x = 2$
			\item $g(x) > 0$ pour $x > 2$
		\end{itemize}
	\end{exemple}
	
	\subsection{Étude du signe d'une fonction polynôme du second degré}
	
	L'étude du signe d'une fonction polynôme du second degré sous forme factorisée se fait en utilisant un tableau de signes.
	
	\begin{propriete}
		Soit $f(x) = a(x - \alpha)(x - \beta)$ avec $\alpha \leq \beta$.
		
		Le signe de $f(x)$ est obtenu en appliquant la règle des signes au produit :
		$$f(x) = a \times (x - \alpha) \times (x - \beta)$$
	\end{propriete}
	
	\subsection{Exemple détaillé}
	
	\begin{exemple}
		Étudions le signe de $f(x) = -3(x - 1)(x + 2)$.
		
		\textbf{Identification :} $a = -3 < 0$, racines : $-2$ et $1$
		
		\textbf{Tableau de signes :}
		
		\begin{center}
			\begin{tabular}{|c|c|c|c|c|c|}
				\hline
				$x$ & $-\infty$ & $-2$ & & $1$ & $+\infty$ \\
				\hline
				$-3$ & $-$ & $-$ & $-$ & $-$ & $-$ \\
				\hline
				$x + 2$ & $-$ & $0$ & $+$ & $+$ & $+$ \\
				\hline
				$x - 1$ & $-$ & $-$ & $-$ & $0$ & $+$ \\
				\hline
				$(x + 2)(x - 1)$ & $+$ & $0$ & $-$ & $0$ & $+$ \\
				\hline
				$f(x) = -3(x + 2)(x - 1)$ & $-$ & $0$ & $+$ & $0$ & $-$ \\
				\hline
			\end{tabular}
		\end{center}
		
		\textbf{Conclusion :}
		\begin{itemize}
			\item $f(x) < 0$ pour $x \in ]-\infty ; -2[ \cup ]1 ; +\infty[$
			\item $f(x) = 0$ pour $x \in \{-2 ; 1\}$
			\item $f(x) > 0$ pour $x \in ]-2 ; 1[$
		\end{itemize}
	\end{exemple}
	
	\section{Relations entre coefficients et racines}
	
	Lorsqu'une fonction polynôme du second degré admet deux racines, il existe des relations importantes entre ces racines et les coefficients de la forme développée.
	
	\begin{propriete}
		Soit $f(x) = ax^2 + bx + c$ une fonction polynôme du second degré admettant deux racines $\alpha$ et $\beta$.
		
		Alors :
		\begin{align*}
			\text{Somme des racines : } \quad \alpha + \beta &= -\frac{b}{a} \\
			\text{Produit des racines : } \quad \alpha \times \beta &= \frac{c}{a}
		\end{align*}
	\end{propriete}
	
	\begin{demonstration}
		Si $f(x) = ax^2 + bx + c$ admet deux racines $\alpha$ et $\beta$, alors on peut écrire :
		$f(x) = a(x - \alpha)(x - \beta)$
		
		Développons la forme factorisée :
		\begin{align*}
			f(x) &= a(x - \alpha)(x - \beta) \\
			&= a(x^2 - \beta x - \alpha x + \alpha \beta) \\
			&= a(x^2 - (\alpha + \beta)x + \alpha \beta) \\
			&= ax^2 - a(\alpha + \beta)x + a\alpha \beta
		\end{align*}
		
		Par identification avec $f(x) = ax^2 + bx + c$ :
		\begin{align*}
			\text{Coefficient de } x : \quad b &= -a(\alpha + \beta) \quad \Rightarrow \quad \alpha + \beta = -\frac{b}{a} \\
			\text{Terme constant : } \quad c &= a\alpha \beta \quad \Rightarrow \quad \alpha \beta = \frac{c}{a}
		\end{align*}
	\end{demonstration}
	
	\begin{exemple}
		Soit $f(x) = 2x^2 - 6x + 4$.
		
		On remarque que $f(1) = 2(1)^2 - 6(1) + 4 = 2 - 6 + 4 = 0$.
		
		Donc $1$ est une racine évidente de $f$.
		
		On a $a = 2$, $b = -6$, $c = 4$.
		
		D'après les relations entre coefficients et racines :
		\begin{align*}
			\alpha + \beta &= -\frac{(-6)}{2} = 3 \\
			\alpha \times \beta &= \frac{4}{2} = 2
		\end{align*}
		
		Comme $\alpha = 1$, on a :
		\begin{align*}
			1 + \beta &= 3 \quad \Rightarrow \quad \beta = 2 \\
			1 \times \beta &= 2 \quad \Rightarrow \quad \beta = 2
		\end{align*}
		
		Les deux relations donnent bien $\beta = 2$. Les racines de $f$ sont donc $1$ et $2$.
		
		\textbf{Vérification :} $f(2) = 2(2)^2 - 6(2) + 4 = 8 - 12 + 4 = 0$ 
	\end{exemple}
	
	\subsection{Application pratique}
	
	\begin{methode}
		Lorsqu'une racine est évidente, on peut utiliser les relations entre coefficients et racines pour trouver l'autre racine rapidement.
		
		\textbf{Méthode :} Si $\alpha$ est une racine connue et $\beta$ l'autre racine :
		\begin{itemize}
			\item $\beta = -\frac{b}{a} - \alpha$ (en utilisant la somme)
			\item Vérifier avec le produit : $\alpha \times \beta = \frac{c}{a}$
			\item on retiendra en particulier que quand une des racines vaut 1, l'autre vaut $\frac{c}{a}$ (car le produit vaut $\frac{c}{a}$)
		\end{itemize}
	\end{methode}
	
\end{document}